
%%%%%%%%%%%%%%%%%%%%%%%%%%%%%%%%%%%%%%%%%%%%%%%%%%%%%%%%%%%%%%%%%%%%%%%%%%%%%%%%
%Definition des Seitenlayout

\frenchspacing                                        	% Gleicht Abstände zwischen Satzzeichen und Worten an

\pagestyle{scrheadings}		
\renewcommand*\chapterpagestyle{scrheadings}			% Header auch auf erste Seite eines Kapitels nutzen + im Inhaltsverzeichnis							
\clearpairofpagestyles									% Defaulteinstellungen für Header-/Footer zurücksetzen


\addtokomafont{pagehead}{\normalfont}					% Header mit greader Schrift (normal wäre die Schrift kursiv)
% Seitenlayout doppelseitig
\lehead{\thepage}                                    	% Kopfzeile links, gerade Seitenzahl (Seitenzahl)
\rehead{\leftmark}                                   	% Kopfzeile rechts, gerade Seitenzahl (Kapitel)
\rohead{\thepage}                                    	% Kopfzeile rechts, ungerade Seitenzahl (Seitenzahl)
\lohead{\leftmark}                                   	% Kopfzeile links, ungerade Seitenzahl (Kapitel)

% Fußzeile Abschalten
\lefoot{}												% Fußzeile links, gerade Seitenzahl (leer)
\lofoot{}												% Fußzeile links, ungerade Seitenzahl (leer)	
\refoot{}												% Fußzeile rechts, gerade Seitenzahl (leer)
\rofoot{}												% Fußzeile recgts, ungerade Seitenzahl (leer)	

%%%%%%%%%%%%%%%%%%%%%%%%%%%%%%%%%%%%%%%%%%%%%%%%%%%%%%%%%%%%%%%%%%%%%%%%%%%%%%%%%%%%%%%%%
%Formelzeichenverzeichnis

% \renewcommand{\nomname}{Formelzeichenverzeichnis}        %Namensänderung von "Nomenclature" zu "Formelzeichenverzeichnis"
\renewcommand{\nomname}{Nomenclature} 
%%%%%%%%%%%%%%%%%%%%%%%%%%%%%%%%%%%%%%%%%%%%%%%%%%%%%%%%%%%%%%%%%%%%%%%%%%%%%%%%
% Literaturverzeichnis

\bibliographystyle{unsrtdin}                						%Literaturangaben nach Erscheinen im Text sortiert, "DIN 1505 Teil 2"

%%%%%%%%%%%%%%%%%%%%%%%%%%%%%%%%%%%%%%%%%%%%%%%%%%%%%%%%%%%%%%%%%%%%%%%%%%%%%%%%
% Zusätzliche Worttrennungen
\hyphenation{Chip-lö-tung}
\hyphenation{Threshold}
\hyphenation{Kol-lek-tor-sät-ti-gungs-span-nung}
\hyphenation{IGBT-Durch-lass-span-nung}
\hyphenation{Ei-gen-er-wär-mung}












														% Falls Latex ein Wort nicht/falsch trennt dies bitte in hyphenation.tex eintragen.

\setlength{\parindent}{0pt}                          	% 1. Zeile nach Absatz einrücken (0pt = nicht einrücken)

\textheight = 690pt                                  	% Textbody vergrößert, Standard:595pt
\voffset = 0.8cm                                     	% Abstand vom oberen Rand der Seite

%%%%%%%%%%%%%%%%%%%%%%%%%%%%%%%%%%%%%%%%%%%%%%%%%%%%%%%%%%%%%%%%%%%%%%%%%%%%%%%%
%Caption-Formatierung
\captionsetup{format=hang}                          % Hängende Captions
\captionsetup{labelfont={bf}}                       % Caption-Bezeichnung ist fett gedruckt
\captionsetup{font={footnotesize}}                  % Caption kleinere Schrifgröße
\captionsetup{margin=1cm}                           % Caption Rand links und rechts
\captionsetup*[table]{position=top}                  % Tabellenbeschriftung oberhalb
\renewcommand{\tablename}{Tabelle}                  % Tabellenbezeichnung wird mit Tab. abgekürzt
\subcaphangtrue                                     % Hängende Subcaptions

%%%%%%%%%%%%%%%%%%%%%%%%%%%%%%%%%%%%%%%%%%%%%%%%%%%%%%%%%%%%%%%%%%%%%%%%%%%%%%%%
%Grafiken
\graphicspath{{grafiken/}}                                   % Verzeichnis für Grafiken
\setlength{\unitlength}{1cm}                                % Einheit für die picture-Umgebung auf 1cm gesetzt

%%%%%%%%%%%%%%%%%%%%%%%%%%%%%%%%%%%%%%%%%%%%%%%%%%%%%%%%%%%%%%%%%%%%%%%%%%%%%%%%
%Zusätzliche Farben
\definecolor{darkblue}{rgb}{0,0,.6}
\definecolor{darkred}{rgb}{.6,0,0}
\definecolor{darkgreen}{rgb}{0,.6,0}

%%%%%%%%%%%%%%%%%%%%%%%%%%%%%%%%%%%%%%%%%%%%%%%%%%%%%%%%%%%%%%%%%%%%%%%%%%%%%%%%
%Listings-Paket
\renewcommand{\lstlistingname}{Quelltext}
\lstset{numbers=left,
        numberstyle=\tiny,
        numbersep=5pt,
        basicstyle=\small,
        breaklines=true,
        keywordstyle=\color{blue},
        commentstyle=\color{darkgreen},
        belowcaptionskip=0.4cm,
        captionpos=b,
        firstnumber=1,
        stepnumber=1,
        frame=leftline,
        rulecolor=\color{black}}

%%%%%%%%%%%%%%%%%%%%%%%%%%%%%%%%%%%%%%%%%%%%%%%%%%%%%%%%%%%%%%%%%%%%%%%%%%%%%%%%
% siuntitx
\sisetup{output-decimal-marker={,}}			% Komma als Dezimaltrennzeichen