%%%%%%%%%%%%%%%%%%%%%%%%%%%%%%%%%%%%%%%%%%%%%%%%%%%%%%%%%%%%%%%%%%%%%%%%%%%%%%%%
%Einbindung von Paketen

%Deutsche Sprache
\usepackage[ngerman]{babel}                                 % Mehrsprachenumgebung Babel mit Deutscher Sprache


%Kodierungen
\usepackage[utf8]{inputenc}                               	% Eingabekodierung & Unterstützung von Umlauten (ä,ö,ü)
\usepackage[T1]{fontenc}                                    % Trennung von Worten mit Umlauten


%Schriftpakete
\usepackage{bm}                                             % Fette Schriftzeichen in der Mathematik-Umgebung
\usepackage{mathptmx}                                       % Times New Roman
\usepackage[scaled=.90]{helvet}                             % Serifenlose Schrift für \textsf
\usepackage{courier}                                        % Schriftart für \texttt
\DeclareSymbolFont{letters}{OML}{cmm}{m}{it}                % Buchstaben der Mathematik-Umgebung in Computer Modern
\DeclareSymbolFont{symbols}{OMS}{cmsy}{m}{n}                % Symbole der Mathematik-Umgebung in Computer Modern
\usepackage{grffile}																				% Ermöglicht Leerzeichen und mehrere Punkte in Pfadangaben


%Symbole
\usepackage{marvosym}                                       % Zusätzliche Symbole (u.A. Euro)
\usepackage{latexsym}                                       % Zusätzliche mathematische Symbole (11)
\usepackage{stmaryrd}                                       % Binäroperatoren


%Grafische Umgebung
\usepackage{color}                                          % Ermöglicht farbige Texte
\usepackage{epic}                                           % Picture-Umgebung: Einbinden von .pic-Grafiken
\usepackage{eepic}                                          % Erweiterung für Picture-Umgebung
\usepackage{epsfig}
\usepackage{graphicx}                                       % Einbinden von Grafiken
\usepackage{subfigure}                                      % Unterabbildungen mit eigenen Unterschriften
\usepackage{pstricks}
\usepackage[section]{placeins} 								% Erlaubt Bereichsbeschränkungen für Float-Objekte (figures) mit \FloatBarrier
															% [section] definiert, dass figures nicht erst in der nächsten section platziert werden dürfen

% Matlab2Tikz
\usepackage{tikz}
\usepackage{tikz}
\usepackage{pgfplots} 																			% https://github.com/matlab2tikz/matlab2tikz
\pgfplotsset{compat=newest} 
\pgfplotsset{plot coordinates/math parser=false} 
\usetikzlibrary{plotmarks} 
\newlength\figureheight 
\newlength\figurewidth 


%Tabellen
\usepackage{longtable}                                      % Paket für Tabellen, die über mehrere Seiten gehen
\usepackage{multicol}                                       % Paket für Text in mehreren Spalten
\usepackage{multirow}					                    					% Paket für Text in mehreren Zeilen
\usepackage{rccol}                                          % Spaltenausrichtung am Komma
\usepackage{booktabs}                                       % Paket für toprule/midrule/bottomrule
\usepackage{hhline}																					% Erlaubt doppelte horizontale Linien \hhline


%Indexerstellung
\usepackage[intoc,german]{nomentbl}                         % Erstellung eines Formelverzeichnisses


%Sonstige Pakete
\usepackage{amsmath}                                 				% Mathematik-Umgebung
\usepackage[bottom]{footmisc}																% Erleichtert Fußnoten in Captions, zwingt Fußnoten an das Ende der Seite (Kann sonst mit Float-Objekten (Bildern) zu Chaos fürhen)
%\usepackage{fancyhdr}                                      % Paket zur Gestaltung von Kopf- und Fußzeile
\usepackage[headsepline]{scrlayer-scrpage}									% Paket zur Gestaltung von Kopf- und Fußzeile
\usepackage{scrhack}																				% Patches...
\usepackage[breaklinks=true, hidelinks]{hyperref}         	% Links in PDf Dokumenten erzeugen
\usepackage{array}                                          % Erstellung von Arrays
\usepackage{setspace}                                       % Paket um Zeilenabstand zu ändern
\usepackage{caption}                                        % Paket für Captions in Tabellen und Bildern
\usepackage[figuresright]{rotating}                         % Paket um Tabellen, Bilder zu drehen (zum rechten Rand gedreht)
\usepackage{listings}                                       % Paket für Quelltexte
\usepackage[framed,numbered]{matlab-prettifier}							% Zusatz zu listings. Ermöglicht Codeeinfärbung identisch zu Matlab
\usepackage{pdfpages} 																			% Einbinden von PDFs
\usepackage{import}																					% Erlaubt relative Pfadangaben
\usepackage{siunitx}              													% Paket für Einheiten
\DeclareSIUnit \var {var}
\usepackage{todonotes}																			% Todo-Notes im Text erstellen
%\usepackage[disable]{todonotes}														% Vor dem Drucken Todo Notes hier global deaktivieren!

