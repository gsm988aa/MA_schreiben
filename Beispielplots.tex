\section*{Beispielplots}

\subsubsection{Gutes Beispiel}

\begin{figure}[!ht]
	\centering
	\setlength\figureheight{7cm}
	\setlength\figurewidth{13cm}
	% This file was created by matlab2tikz.
%
%The latest updates can be retrieved from
%  http://www.mathworks.com/matlabcentral/fileexchange/22022-matlab2tikz-matlab2tikz
%where you can also make suggestions and rate matlab2tikz.
%
\begin{tikzpicture}

\begin{axis}[%
width=\figurewidth,
height=0.907\figureheight,
at={(0\figurewidth,0\figureheight)},
scale only axis,
separate axis lines,
every outer x axis line/.append style={black},
every x tick label/.append style={font=\color{black}},
xmin=0,
xmax=6,
xtick={0,1,2,3,4,5,6},
xlabel={Zeit in s},
xmajorgrids,
every outer y axis line/.append style={black},
every y tick label/.append style={font=\color{black}},
ymin=-0.6,
ymax=0.6,
ytick={-0.6,-0.4,-0.2,0,0.2,0.4,0.6},
yticklabels={{-0,6},{-0,4},{-0,2},{0},{0,2},{0,4},{0,6}},
ylabel={Strom in A},
ymajorgrids,
axis background/.style={fill=white},
legend style={legend cell align=left,align=left,draw=white!15!black}
]
\addplot [color=black,solid,line width=1.0pt]
  table[row sep=crcr]{%
0	0\\
0.1	0.109268131937284\\
0.2	0.214180088269758\\
0.3	0.31055336036727\\
0.4	0.394545849994738\\
0.5	0.462809041644343\\
0.6	0.512621497281975\\
0.7	0.541997351493653\\
0.8	0.549765481672828\\
0.9	0.535616196983007\\
1	0.500113584754125\\
1.1	0.444673022100775\\
1.2	0.371504749303133\\
1.3	0.283525754501805\\
1.4	0.184243482585748\\
1.5	0.077616004432927\\
1.6	-0.0321057788851691\\
1.7	-0.140547606114757\\
1.8	-0.243386243812169\\
1.9	-0.336521840018496\\
2	-0.416241372419361\\
2.1	-0.479366674827474\\
2.2	-0.523381140639234\\
2.3	-0.546530051998406\\
2.4	-0.547890534859712\\
2.5	-0.527408351064726\\
2.6	-0.485900060646084\\
2.7	-0.425020468155793\\
2.8	-0.347196650829777\\
2.9	-0.255531198677566\\
3	-0.153678524009409\\
3.1	-0.045699171549623\\
3.2	0.0641020626677715\\
3.3	0.171347749932358\\
3.4	0.271762343126234\\
3.5	0.361342629295334\\
3.6	0.436517325117034\\
3.7	0.494289452696395\\
3.8	0.532355819617317\\
3.9	0.549198839956033\\
4	0.54414703564286\\
4.1	0.517401806173875\\
4.2	0.470029399448554\\
4.3	0.403918403830762\\
4.4	0.321704456090469\\
4.5	0.226665166882966\\
4.6	0.122589452755136\\
4.7	0.0136264839993468\\
4.8	-0.0958797296726388\\
4.9	-0.201563521088561\\
5	-0.299211610989153\\
5.1	-0.384931078176448\\
5.2	-0.45530455799711\\
5.3	-0.507526481887044\\
5.4	-0.53951492653657\\
5.5	-0.549994613602887\\
5.6	-0.538547751033225\\
5.7	-0.505630689115572\\
5.8	-0.45255572723279\\
5.9	-0.381438796627417\\
6	-0.295115104900239\\
6.1	-0.197026105230255\\
6.2	-0.0910822964965702\\
};
\addlegendentry{$i_1$};

\addplot [color=blue,dashed,line width=1.0pt]
  table[row sep=crcr]{%
0	0.242487113059643\\
0.1	0.27992203730887\\
0.2	0.248822284202539\\
0.3	0.156802157942887\\
0.4	0.0263913947523759\\
0.5	-0.110480902305586\\
0.6	-0.220303621322969\\
0.7	-0.276188330483089\\
0.8	-0.264452503936179\\
0.9	-0.187969481322183\\
1	-0.0654649740156728\\
1.1	0.0730676421006614\\
1.2	0.193710751107647\\
1.3	0.266926712344854\\
1.4	0.274789705005495\\
1.5	0.21537459425479\\
1.6	0.103228271378942\\
1.7	-0.034191932542297\\
1.8	-0.163240758891846\\
1.9	-0.252322554244172\\
2	-0.279626988260801\\
2.1	-0.238468983219033\\
2.2	-0.138925454188702\\
2.3	-0.00536812877837581\\
2.4	0.129503501776933\\
2.5	0.232668158504726\\
2.6	0.278867535444852\\
2.7	0.25679041386277\\
2.8	0.171842043088305\\
2.9	0.0448207469650515\\
3	-0.0931742311934444\\
3.1	-0.20835690799087\\
3.2	-0.272526547010924\\
3.3	-0.269972182627098\\
3.4	-0.201319212327125\\
3.5	-0.0833762775964814\\
3.6	0.0549800777391185\\
3.7	0.179875392546936\\
3.8	0.260730937885635\\
3.9	0.277750456320574\\
4	0.226766976162424\\
4.1	0.120263031464932\\
4.2	-0.0156854976550285\\
4.3	-0.147793669898183\\
4.4	-0.243716797265827\\
4.5	-0.279969552742338\\
4.6	-0.24767599742802\\
4.7	-0.154742719940931\\
4.8	-0.0239230277712742\\
4.9	0.112753855941553\\
5	0.221824663291686\\
5.1	0.276585056662421\\
5.2	0.263627781921117\\
5.3	0.1861252318252\\
5.4	0.0630527336340797\\
5.5	-0.075457272791626\\
5.6	-0.195492707173551\\
5.7	-0.267664708792873\\
5.8	-0.274303054566629\\
5.9	-0.213782445929001\\
6	-0.100920438604497\\
6.1	0.0366504118137338\\
6.2	0.165247963192162\\
};
\addlegendentry{$i_2$};

\end{axis}
\end{tikzpicture}%
	\caption{Verlauf der Str\"ome auf der Primär- und der Sekundärseite in Abh\"angigkeit von der Zeit} 
	\label{abb:stromverlauf_gutes_bsp}
\end{figure}


\subsubsection{Schlechtes Beispiel}

\begin{figure}[!ht]
	\centering
	\setlength\figureheight{7cm}
	\setlength\figurewidth{14cm}
	% This file was created by matlab2tikz.
%
%The latest updates can be retrieved from
%  http://www.mathworks.com/matlabcentral/fileexchange/22022-matlab2tikz-matlab2tikz
%where you can also make suggestions and rate matlab2tikz.
%
\definecolor{mycolor1}{rgb}{0.00000,1.00000,1.00000}%
%
\begin{tikzpicture}

\begin{axis}[%
width=\figurewidth,
height=0.907\figureheight,
at={(0\figurewidth,0\figureheight)},
scale only axis,
separate axis lines,
every outer x axis line/.append style={black},
every x tick label/.append style={font=\color{black}},
xmin=0,
xmax=7,
xlabel={Zeit in $s$~\Huge\textcolor{red}{7}},
xmajorgrids,
every outer y axis line/.append style={black},
every y tick label/.append style={font=\color{black}},
ymin=-0.5,
ymax=0.5,
ytick={-0.4, -0.2,    0,  0.2,  0.4},
ylabel={Strom [A]~\Huge\textcolor{red}{1}},
ymajorgrids,
axis background/.style={fill=white},
legend style={at={(0.03,0.97)},anchor=north west,legend cell align=left,align=left,draw=white!15!black}
]
\addplot [color=blue,solid,line width=1.0pt]
  table[row sep=crcr]{%
0	0\\
0.1	0.0894011988577775\\
0.2	0.175238254038893\\
0.3	0.254089113027766\\
0.4	0.322810240904785\\
0.5	0.378661943163553\\
0.6	0.419417588685252\\
0.7	0.443452378494807\\
0.8	0.449808121368677\\
0.9	0.438231433895188\\
1	0.409183842071557\\
1.1	0.363823381718816\\
1.2	0.303958431248018\\
1.3	0.231975617319659\\
1.4	0.150744667570157\\
1.5	0.0635040036269403\\
1.6	-0.026268364542411\\
1.7	-0.114993495912074\\
1.8	-0.199134199482684\\
1.9	-0.275336050924224\\
2	-0.340561122888568\\
2.1	-0.392209097586115\\
2.2	-0.428220933250282\\
2.3	-0.447160951635059\\
2.4	-0.448274073976128\\
2.5	-0.431515923598412\\
2.6	-0.397554595074069\\
2.7	-0.347744019400194\\
2.8	-0.284069987042544\\
2.9	-0.20907098073619\\
3	-0.125736974189517\\
3.1	-0.0373902312678734\\
3.2	0.0524471421827221\\
3.3	0.14019361358102\\
3.4	0.222351008012374\\
3.5	0.295643969423455\\
3.6	0.357150538732119\\
3.7	0.404418643115232\\
3.8	0.435563852414169\\
3.9	0.449344505418572\\
4	0.445211210980522\\
4.1	0.423328750505898\\
4.2	0.384569508639726\\
4.3	0.330478694043351\\
4.4	0.263212736801293\\
4.5	0.18545331835879\\
4.6	0.100300461345111\\
4.7	0.011148941454011\\
4.8	-0.0784470515503408\\
4.9	-0.164915608163368\\
5	-0.244809499900216\\
5.1	-0.314943609417094\\
5.2	-0.372521911088544\\
5.3	-0.415248939725763\\
5.4	-0.441421303529921\\
5.5	-0.449995592947817\\
5.6	-0.440629978118093\\
5.7	-0.413697836549104\\
5.8	-0.370272867735919\\
5.9	-0.312086288149705\\
6	-0.241457813100196\\
6.1	-0.161203177006572\\
6.2	-0.0745218789517392\\
};
\addlegendentry{i$_1$};

\addplot [color=mycolor1,solid,line width=1.0pt]
  table[row sep=crcr]{%
0	0.242487113059643\\
0.1	0.27992203730887\\
0.2	0.248822284202539\\
0.3	0.156802157942887\\
0.4	0.0263913947523759\\
0.5	-0.110480902305586\\
0.6	-0.220303621322969\\
0.7	-0.276188330483089\\
0.8	-0.264452503936179\\
0.9	-0.187969481322183\\
1	-0.0654649740156728\\
1.1	0.0730676421006614\\
1.2	0.193710751107647\\
1.3	0.266926712344854\\
1.4	0.274789705005495\\
1.5	0.21537459425479\\
1.6	0.103228271378942\\
1.7	-0.034191932542297\\
1.8	-0.163240758891846\\
1.9	-0.252322554244172\\
2	-0.279626988260801\\
2.1	-0.238468983219033\\
2.2	-0.138925454188702\\
2.3	-0.00536812877837581\\
2.4	0.129503501776933\\
2.5	0.232668158504726\\
2.6	0.278867535444852\\
2.7	0.25679041386277\\
2.8	0.171842043088305\\
2.9	0.0448207469650515\\
3	-0.0931742311934444\\
3.1	-0.20835690799087\\
3.2	-0.272526547010924\\
3.3	-0.269972182627098\\
3.4	-0.201319212327125\\
3.5	-0.0833762775964814\\
3.6	0.0549800777391185\\
3.7	0.179875392546936\\
3.8	0.260730937885635\\
3.9	0.277750456320574\\
4	0.226766976162424\\
4.1	0.120263031464932\\
4.2	-0.0156854976550285\\
4.3	-0.147793669898183\\
4.4	-0.243716797265827\\
4.5	-0.279969552742338\\
4.6	-0.24767599742802\\
4.7	-0.154742719940931\\
4.8	-0.0239230277712742\\
4.9	0.112753855941553\\
5	0.221824663291686\\
5.1	0.276585056662421\\
5.2	0.263627781921117\\
5.3	0.1861252318252\\
5.4	0.0630527336340797\\
5.5	-0.075457272791626\\
5.6	-0.195492707173551\\
5.7	-0.267664708792873\\
5.8	-0.274303054566629\\
5.9	-0.213782445929001\\
6	-0.100920438604497\\
6.1	0.0366504118137338\\
6.2	0.165247963192162\\
};
\addlegendentry{$i_2$};
\node[font=\bfseries] at (0.2,-0.42) {\Huge\textcolor{red}{2}};
\node[font=\bfseries] at (1,0.45) {\Huge\textcolor{red}{3}};
\node[font=\bfseries] at (1,0.34) {\Huge\textcolor{red}{4}};
\node[font=\bfseries] at (6.5,0.4) {\Huge\textcolor{red}{5}};
\node[font=\bfseries] at (6.5,0) {\Huge\textcolor{red}{6}};
\node[font=\bfseries] at (2.2,0.07) {\Huge\textcolor{red}{9}};
\end{axis}
\end{tikzpicture}%
	\caption{Str\"ome in Abh\"angigkeit der Zeit~\Huge\textcolor{red}{8}} 
	\label{abb:stromverlauf_schlechtes_bsp}
\end{figure}

\newpage


\subsubsection{Erläuterungen zum Plot}

\begin{enumerate}
	\item[1)] Einheiten nicht in eckige Klammer schreiben. Richtig ist:
			\begin{itemize}
				\item Strom in A
				\item Strom /A
				\item Strom $i$ in A
				\item Strom $i$ /A
				\item $i$ in A
				\item $i$ /A
				\item Wichtig ist: immer eine einheitliche Darstellung verwenden
				\item zur Info: diese Schreibweise ist korrekt: [$i$]~=~A, wird aber nicht als Achsenbeschriftung verwendet
			\end{itemize}
	\item[2)] In deutschen Arbeiten Dezimalkomma verwenden. In englischen Veröffentlichungen Dezimalpunkte verwenden.
	\item[3)] Einheiten sind kursiv zu schreiben, Indizes gerade (es sei denn, der Index ist eine Variable, z.B. Zählvariable)
	\item[4)] Legende muss an einer günstigen Stelle platziert sein, möglichst Verdeckung vermeiden
	\item[5)] Abbildung darf nicht über den Rand des Dokuments ragen
	\item[6)] Verläufe sollten bis zum Ende des Plots gehen. Die Achse muss nicht zwingend mit einem Zahlenwert abgeschlossen werden.
	\item[7)] Einheiten immer gerade schreiben, nicht kursiv
	\item[8)] Es heißt immer: in Abhängigkeit \textbf{von} ...
	\item[9)] Geeignete Plotfarben wählen, die gut erkennbar sind. Für s/w-Ausdrucke ist es sinnvoll auch die Art des Plots zu unterscheiden (durchgezogen, gestrichelt, gepunktet, markiert)
\end{enumerate}

\subsubsection{Tipps}

\begin{itemize}
	\item Bildunterschrift muss vollständig und eindeutig sein
	\item Möglichst nicht mehr als drei bis vier Verläufe pro Plot abbilden
	\item Skalierung der Achsen sinnvoll wählen
\end{itemize}